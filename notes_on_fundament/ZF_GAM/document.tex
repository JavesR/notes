\documentclass[11pt,a4paper]{article}
\usepackage{amsmath}
\usepackage{amssymb}
\usepackage{graphicx}
\usepackage{subfigure}
\usepackage{float}
\usepackage{xeCJK}
\usepackage{geometry}
\geometry{left=2.0cm,right=2.0cm,top=2.0cm,bottom=2.0cm}

\title{zonal flow/geodesic acoustic mode}
\author{Guangzhi Ren}
\date{\today}

\begin{document}
	
	\maketitle
	
\section{zonal flow}


\newpage
\section{residula flow}	

	\subsection{history}
	\begin{itemize}
		\item gyrofluid simulation [Beer,Waltz]: a total collisionless decay of poloidal rotation
		\item gyrokinetic analysis by [RH 1998]: linear collisionless kinetic mechanisms do not damp the zonal flows completely
		\item verification by various gyrokinetic codes
		\item considering shaping effect [Xiao 2006]
		\item modification of zonal flow closures in gyrofluid model[Mandell,Sugama]
	\end{itemize}
	
	\subsection{description}
	Considering the polarization drift in plasma, 
\begin{equation}
	v_{pj}=\frac{m_j}{e_jB^2}\frac{d\pmb{E}}{dt}
	\end{equation}
	this gives rise to the polarization current,
	\begin{equation}
	j^{cl}=\sum_{j}{e_jn_j\pmb{v}_{pj}}=\epsilon_0\epsilon^{cl}\frac{d\pmb{E}}{dt}
	\end{equation}
	where $\epsilon^{cl}=m_in_i/\epsilon_0B^2=(\omega_{pi}/\Omega_i)^2=(k_{Di}\rho_i)^2>>1$. This polarization current originates from delayed ion gyro motion from time varying electric field and $\epsilon^{cl}$ is called classical polarization, which is a low frenquency dielectric constant perpendicular to the magnetic field.

	And in tokamak, considering the toroidal effect, we should include the traped and passing particals. Some fraction of charged particals($f_t\sim\sqrt{\epsilon},\epsilon=r/R$) are trapped by the magnetic mirror and have a radial excursion by 
	\begin{equation}
		\Delta_t=\sqrt{\epsilon}\rho_{pi}=\frac{q\rho_i}{\sqrt{\epsilon}}
	\end{equation}
	and as for passing partical the excursion is
	\begin{equation}
		\Delta_p=q\rho_i
	\end{equation}
	
	So during this trapped partial orbit motion, we alse have the similar polarization effect if we have radial electric field $E_r$, 
	\begin{equation}
		j^{nc}=\epsilon_0\epsilon^{nc}\frac{dE_r}{dt}
	\end{equation}
	\begin{equation}
		\epsilon^{nc}=\sqrt{\epsilon}k_{Di}^2\Delta_t^2=\frac{q^2}{\sqrt{\epsilon}}\epsilon^{cl}
	\end{equation}
	and Hinton-Rosenbluth give a expression for the polarization as,
	\begin{equation}
		\epsilon=\epsilon^{cl}+\epsilon^{nc}=\frac{\omega_{pi}^2}{\Omega_i^2}(1+\frac{1.6q^2}{\sqrt{\epsilon}})
	\end{equation}
	The factor 1.6 comes from detialed kinetic calculation including passing partical contribution.
	
	Then we consider the continuity equation of the polarization current,
	\begin{equation}
		\frac{\partial\rho_p}{\partial{t}}+\nabla\cdot\pmb{j}_p=S		
	\end{equation}
	$S$ is the external source density.Taking the flux surface average and Fourier expansion in space, 
	\begin{equation}
		\frac{\partial}{\partial{t}}<\rho_p(\pmb{k})>+<i\pmb{k}_\perp\cdot{\pmb{j}_p}(\pmb{k})>=<S(\pmb{k})>
	\end{equation}
	\begin{equation}
		\epsilon_0\epsilon_p<k_\perp^2>\Phi_k=<\rho_p>-\int<S(\pmb{k})>dt
	\end{equation}
	Consider an initial source perturbation $<S(\pmb{k})>=\delta_k(0)\delta(t)$. when the time scale is in a few gyro motion and much shorter than the bounce time of trapped partical.we have,
	\begin{equation}
		\epsilon_0\epsilon^{cl}\Phi_k(t=+0)=-e_i\delta{n_k(0)}
	\end{equation} 
	when the time scale is longer than the bounce time of trapped particals, the electrostatic potential is further shielded by the addition of the neoclassical polarization, then we have,
	\begin{equation}
		\epsilon_0(\epsilon^{cl}+\epsilon^{nc})\Phi_k(t=+\infty)=-e_i\delta{n_k(0)}
	\end{equation}
	Therefore, the ratio of the long term zonal flow potential to the initial zonal flow potential is given by, 
	\begin{equation}
		\frac{\Phi_k(t=\infty)}{\Phi_k(t=0)}=\frac{\epsilon^{cl}}{\epsilon^{cl}+\epsilon^{nc}}
	\end{equation}
	Using the Hinton formula, we have,
	\begin{equation}
		\frac{\Phi_k(t=\infty)}{\Phi_k(t=0)}=\frac{1}{1+1.6q^2/\sqrt{\epsilon}}
	\end{equation}
		
	
	[reference]
	\begin{enumerate}
		\item Kikuchi ebook
		\item Diamond PPCF 2005
		\item Hinton PRL 1998
		\item Idomura ... 
	\end{enumerate}
	
	
\newpage	
\section{Geodesic acoustic mode}
\subsection{history}
	\begin{itemize}
		\item first prediction by [Winsor 1968]
		\item Fully forgotten between 1968 and 1996
		\item dispersion relation, frequency, radial structure and propagation
		\item close relation to alfven eigen mode
		\item experimental observations of GAM, H1-heliac,[Shats PRL 2002]; DIII-
		D, [McKee PoP 2003]
	\end{itemize}
	
\subsection{description}
	\subsubsection{Winsor1968}
	The starting equations are the follows:
	\begin{equation}
	\begin{aligned}
		&\rho_0\frac{\partial{\pmb{v}}}{\partial{t}}=\frac{1}{c}\pmb{J}\times\pmb{B}-\nabla{p}	\\
		&\frac{\partial\rho}{\partial{t}}+\nabla\cdot\rho_0\pmb{v}=0	\\
		&\nabla\phi=\frac{1}{c}\pmb{b}\times\pmb{B}	\\
		&\nabla\cdot\pmb{J}=0	\\
		&\rho_0^{-\gamma}-\gamma{p_0}\rho_0^{-\gamma-1}\frac{\partial{\rho}}{\partial{t}}+\pmb{v}\cdot\nabla({p_0}\rho_0^{-\gamma})=0
	\end{aligned}
	\end{equation}	
	finally, we can deduce the dispersion relation:
	\begin{equation}
		\omega^2\int{|\rho|^2}\mathcal{J}dS=\frac{\gamma{p_0}}{\rho_0}
		( |\int\rho_0\frac{\pmb{B}\times\nabla\psi\cdot\nabla{B^2}}{B^4}\mathcal{J}dS|^2/\int\frac{|\nabla\psi|^2}{B^2}\mathcal{J}dS + \int\frac{|\pmb{B}\cdot\nabla\rho|^2}{B^2}\mathcal{J}dS )
	\end{equation}
	The first term is due to motion in the $\pmb{B}\times\nabla\psi$direction.It is associated with geodesic curvature, i.e., the surface component of the magnetic field line curvature. And the second \textbf{ordinary sound propagation propagating along the field lines}.
	
	The physical mechanism of geodesic acoustic mode is explained as follows. An electric field pertubation $E_\psi$ causes a flow $\pmb{v_\perp}=\pmb{E}\times\pmb{B}/B^2$ and since the compressibility, the flow causes a density accumulation proportional to $-\nabla\cdot\pmb{v_\perp}=\pmb{E}\times\pmb{B}\cdot\nabla{B^2}/B^4$.This $n$ generates a current $\pmb{J}=\pmb{B}\times\nabla{n}/B^2$ which transports charge across the magnetic surface and acts to recerse $E_\psi$
	
	In the limit of circular cross section with large aspect ratio($r\ll R$), the dispersion relation becomes,
	\begin{equation}
		\omega^2=\frac{2\gamma{p_0}}{\rho_0{R}^2}[1+1/(2q^2)]=2\frac{C_s^2}{R^2}(1+1/(2q^2))
	\end{equation}
	with the defination of sound wave velocity in neutral gas $C_s=(\gamma{p_0}/\rho_0)^{1/2}$ .
	
	
	\subsubsection{GAM in 5-field landau fluid model}
	
	
	
	
	[reference]
	\begin{enumerate}
		\item Diamond PPCF 2005
		\item Winsor POF 1968
	\end{enumerate}
	
	
	
\end{document}