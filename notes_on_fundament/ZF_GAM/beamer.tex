\documentclass{beamer} 
%\usepackage{xeCJK} % 中文包
\usetheme{Boadilla}  
\usecolortheme{seagull} 

%\setbeamertemplate{caption}[numbered] % 序号
%\setbeamercovered{transparent=15}
\setbeamersize{text margin left=0.6cm, text margin right=0.6cm} % 设置页边距
\setbeamercovered{transparent} 
\setbeamertemplate{navigation symbols}{} %Navigationsleiste ausschalten     


\title{Weekly work summary}
\author{Guangzhi Ren}
\institute {}
\date{\today}

%============================================================
\begin{document}

\begin{frame} %beamer里重要的概念,每个frame定义一张page
\titlepage   
\end{frame}

\begin{frame}{content}
	\begin{itemize}
		\item ZF/RHF/GAM
		\item references on electromagnetic turbulence
		\item modification of energy transfer equation
		\item revisiting to former simulation result 
		\item comparision between different algorithm
	\end{itemize}
\end{frame}

\begin{frame}{Residual zonal flow}
	\begin{itemize}
		\item gyrofluid simulation [Beer,Waltz]: a total collisionless decay of poloidal rotation
		\item gyrokinetic analysis by [RH 1998]: linear collisionless kinetic mechanisms do not damp the zonal flows completely
		\item verification by various gyrokinetic codes
		\item considering shaping effect [Xiao 2006]
		\item modification of zonal flow closures in gyrofluid model[Mandell,Sugama]
	\end{itemize}
\end{frame}

\begin{frame}{Residual zonal flow }
	Considering the polarization drift in plasma, 
	\begin{equation}
		v_{pj}=\frac{m_j}{e_jB^2}\frac{d\pmb{E}}{dt}
	\end{equation}
	this gives rise to the polarization current,
	\begin{equation}
		j^{cl}=\sum_{j}{e_jn_j\pmb{v}_{pj}}=\epsilon_0\epsilon^{cl}\frac{d\pmb{E}}{dt}
	\end{equation}
	where $\epsilon^{cl}=m_in_i/\epsilon_0B^2=(\omega_{pi}/\Omega_i)^2=(k_{Di}\rho_i)^2>>1$
\end{frame}

\begin{frame}{Residula zonal flow}
	And in tokamak, considering the toroidal effect, we should include the traped and passing particals. Some fraction of charged particals($f_t\sim\sqrt{\epsilon},\epsilon=r/R$) are trapped by the magnetic mirror and have a radial excursion by 
	\begin{equation}
	\Delta_t=\sqrt{\epsilon}\rho_{pi}=\frac{q\rho_i}{\sqrt{\epsilon}}
	\end{equation}
	and as for passing partical the excursion is
	\begin{equation}
	\Delta_p=q\rho_i
	\end{equation}
\end{frame}


\begin{frame}{Residula zonal flow}
	So during this trapped partial orbit motion, we alse have the similar polarization effect if we have radial electric field $E_r$, 
	\begin{equation}
	j^{nc}=\epsilon_0\epsilon^{nc}\frac{dE_r}{dt}
	\end{equation}
	\begin{equation}
	\epsilon^{nc}=\sqrt{\epsilon}k_{Di}^2\Delta_t^2=\frac{q^2}{\sqrt{\epsilon}}\epsilon^{cl}
	\end{equation}
	and Hinton-Rosenbluth give a expression for the polarization as,
	\begin{equation}
	\epsilon=\epsilon^{cl}+\epsilon^{nc}=\frac{\omega_{pi}^2}{\Omega_i^2}(1+\frac{1.6q^2}{\sqrt{\epsilon}})
	\end{equation}
	The factor 1.6 comes from detialed kinetic calculation including passing partical contribution.
\end{frame}

\begin{frame}{Residula zonal flow}
	Then we consider the continuity equation of the polarization current,
	\begin{equation}
	\frac{\partial\rho_p}{\partial{t}}+\nabla\cdot\pmb{j}_p=S		
	\end{equation}
	$S$ is the external source density.Taking the flux surface average and Fourier expansion in space, 
	\begin{equation}
	\frac{\partial}{\partial{t}}<\rho_p(\pmb{k})>+<i\pmb{k}_\perp\cdot{\pmb{j}_p}(\pmb{k})>=<S(\pmb{k})>
	\end{equation}
	\begin{equation}
	\epsilon_0\epsilon_p<k_\perp^2>\Phi_k=<\rho_p>-\int<S(\pmb{k})>dt
	\end{equation}
\end{frame}


\begin{frame}{Residula zonal flow}
	Consider an initial source perturbation $<S(\pmb{k})>=\delta_k(0)\delta(t)$. when the time scale is in a few gyro motion and much shorter than the bounce time of trapped partical.we have,
	\begin{equation}
	\epsilon_0\epsilon^{cl}\Phi_k(t=+0)=-e_i\delta{n_k(0)}
	\end{equation} 
	when the time scale is longer than the bounce time of trapped particals, the electrostatic potential is further shielded by the addition of the neoclassical polarization, then we have,
	\begin{equation}
	\epsilon_0(\epsilon^{cl}+\epsilon^{nc})\Phi_k(t=+\infty)=-e_i\delta{n_k(0)}
	\end{equation}
	Therefore, the ratio of the long term zonal flow potential to the initial zonal flow potential is given by, 
	\begin{equation}
	\frac{\Phi_k(t=\infty)}{\Phi_k(t=0)}=\frac{\epsilon^{cl}}{\epsilon^{cl}+\epsilon^{nc}}
	\end{equation}
	Using the Hinton formula, we have,
	\begin{equation}
	\frac{\Phi_k(t=\infty)}{\Phi_k(t=0)}=\frac{1}{1+1.6q^2/\sqrt{\epsilon}}
	\end{equation}
\end{frame}


\begin{frame}{Geodesic acoustic mode}
	\begin{itemize}
		\item first prediction by [Winsor 1968]
		\item Fully forgotten between 1968 and 1996
		\item dispersion relation, frequency, radial structure and propagation
		\item close relation to alfven eigen mode
		\item experimental observations of GAM, H1-heliac,[Shats PRL 2002]; DIII-
		D, [McKee PoP 2003]
	\end{itemize}
\end{frame}

\begin{frame}{Geodesic acoustic mode}
	The starting equations are the follows:
	\begin{equation}
	\begin{aligned}
	&\rho_0\frac{\partial{\pmb{v}}}{\partial{t}}=\frac{1}{c}\pmb{J}\times\pmb{B}-\nabla{p}	\\
	&\frac{\partial\rho}{\partial{t}}+\nabla\cdot\rho_0\pmb{v}=0	\\
	&\nabla\phi=\frac{1}{c}\pmb{b}\times\pmb{B}	\\
	&\nabla\cdot\pmb{J}=0	\\
	&\rho_0^{-\gamma}-\gamma{p_0}\rho_0^{-\gamma-1}\frac{\partial{\rho}}{\partial{t}}+\pmb{v}\cdot\nabla({p_0}\rho_0^{-\gamma})=0
	\end{aligned}
	\end{equation}	
\end{frame}

\begin{frame}{Geodesic acoustic mode}
	finally, we can deduce the dispersion relation:
	\begin{equation}
	\begin{aligned}
		& \omega^2\int{|\rho|^2}\mathcal{J}dS= \\ 
		& \frac{\gamma{p_0}}{\rho_0}( |\int\rho_0\frac{\pmb{B}\times\nabla\psi\cdot\nabla{B^2}}{B^4}\mathcal{J}dS|^2/\int\frac{|\nabla\psi|^2}{B^2}\mathcal{J}dS + \int\frac{|\pmb{B}\cdot\nabla\rho|^2}{B^2}\mathcal{J}dS )
	\end{aligned}
	\end{equation}
	The first term is due to motion in the $\pmb{B}\times\nabla\psi$direction.It is associated with geodesic curvature, i.e., the surface component of the magnetic field line curvature. And the second \textbf{ordinary sound propagation propagating along the field lines}.
\end{frame}


\begin{frame}{Geodesic acoustic mode}
	In the limit of circular cross section with large aspect ratio($r\ll R$), the dispersion relation becomes,
	\begin{equation}
	\omega^2=\frac{2\gamma{p_0}}{\rho_0{R}^2}[1+1/(2q^2)]=2\frac{C_s^2}{R^2}(1+1/(2q^2))
	\end{equation}
	with the defination of sound wave velocity in neutral gas $C_s=(\gamma{p_0}/\rho_0)^{1/2}$ .
	
	\begin{quote}
		{\color{magenta} question: relation between RHF and GAM}
	\end{quote}

\end{frame}


\begin{frame}{electromagnetic turbulence}
	[scott][naliun][may..]
\end{frame}


\begin{frame}{modification of energy transfer equation}
	\begin{equation}
	<f>=\frac{\int{f}dV}{\int{dV}}
	=\frac{\int{f}\ rdrd\theta{d\zeta}}{\int\ rdrd\theta{d\zeta}}
	\end{equation}

	for zonal flow energy $E_k=\frac{1}{2}<v_E^2>$:
	\begin{equation}
	\frac{\partial}{\partial{t}}E_k=\Gamma_{NLk,k}+\Gamma_{NLm,k}+\Gamma_{C,k}
	\end{equation}
	for pressure $E_p=\frac{1}{2}<p^2>$:
	\begin{equation}
	p=p_i+p_e=n_0T+(1+\tau)T_0{n}
	\end{equation}
	
	\begin{equation}
	\frac{\partial}{\partial{t}}E_p=\Gamma_{Q,p}+\Gamma_{F,p}+\Gamma_{C,p}+\Gamma_{LD,p}+\Gamma_{NL,p}+\Gamma_{S,p}
	\end{equation}
\end{frame}

\begin{frame}{modification of energy transfer equation}
	flux surface average
	\begin{equation}
		<f>=\frac{\int{f}dS}{\int{dS}}
		=\frac{\int{f}\ d\theta{d\zeta}}{\int\ d\theta{d\zeta}}
	\end{equation}
	for zonal flow 
	\begin{equation}
		\frac{\partial}{\partial{t}}<v_E>
		=-\frac{1}{r^2}\frac{\partial}{\partial{r}}r^2<v_\theta{v_r}>
		+\frac{\beta}{n_{eq}}\frac{1}{r^2}\frac{\partial}{\partial{r}}r^2<B_\theta{B_r}>
		-\frac{2\epsilon}{n_{eq}}<p\sin\theta>
	\end{equation}
	and in the code, we use the alternative expression,
	\begin{equation}
	\frac{\partial}{\partial{t}}<v_E>
	=\frac{1}{r}<\phi\frac{\partial{\Omega}}{\partial\theta}>
	+\frac{\beta}{n_{eq}}\frac{1}{r}<A\frac{\partial{A}}{\partial\theta}>
	-\frac{2\epsilon}{n_{eq}}<p\sin\theta>
	\end{equation} 
	with the Fourier expansion,
	\begin{equation}
	\frac{\partial}{\partial{t}}<v_E>
	=\sum_{m1+m2=0}{\phi_{m1}}{\Omega_{\theta,m2}}
	+\sum_{m1+m2=0}{A_{m1}}{j_{\theta,m2}}
	-\frac{i\epsilon}{n_{eq}}(p_{1,0}-p_{-1,0})
	\end{equation} 
\end{frame}
	
	
\begin{frame}{modification of energy transfer equation}
	as for equation for thr evolution of $<p\sin\theta>$
	\begin{equation}
		\frac{\partial}{\partial{t}}<p\sin\theta>
		=<(R_Q+R_F+R_C+R_{NL}+R_{LD}+R_{S})\sin\theta>
	\end{equation}

	\begin{equation}
	\begin{aligned}
		<R_Q\sin\theta>&=
		a[(1+\tau)T_{eq}\frac{\partial{n_{eq}}}{\partial{r}}+n_{eq}\frac{\partial{T_{eq}}}{\partial{r}}]<\phi\cos\theta>\\
		&+(1+\tau)\beta{T_{eq}}\frac{\partial{j_0}}{\partial{r}}<A\cos\theta>\\
		<R_F\sin\theta>&=
		(\Gamma+\tau)n_{eq}T_{eq}\frac{\epsilon}{q}<v\cos\theta>
		-(1+\tau)T_{eq}\frac{\epsilon}{q}<j\cos\theta>	\\
		<R_C\sin\theta>&=
		\epsilon(\Gamma+\tau)p_{eq}<v_\theta>	\\
		&+\epsilon(2\Gamma-1)p_{eq}<\frac{\partial{T}}{\partial{r}}>
		+\epsilon((\Gamma-1)-\tau(\tau+1))T_{eq}^2<\frac{\partial{n}}{\partial{r}}>	\\
		<R_{LD}\sin\theta>&=
		-(\Gamma-1)\sqrt{\frac{8T_{eq}}{\pi}}n_{eq}\frac{\epsilon}{q}<T\sin\theta>	\\
	\end{aligned}
	\end{equation}
\end{frame}	
	


\begin{frame}{revisiting to former simulation result}
	...
\end{frame}


\begin{frame}{comparision between different algorithm}
	
\end{frame}





\end{document}