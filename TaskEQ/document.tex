\documentclass[11pt,a4paper]{article}
\usepackage{amsmath}
\usepackage{amssymb}
\usepackage{graphicx}
\usepackage{subfigure}
\usepackage{float}
\usepackage{xeCJK}
\usepackage{geometry}
\geometry{left=2.0cm,right=2.0cm,top=2.0cm,bottom=2.0cm}

\usepackage[T1]{fontenc}
\usepackage{xcolor}
\usepackage{lmodern}
\usepackage{listings}
\definecolor{mygreen}{rgb}{0,0.6,0}
\definecolor{mygray}{rgb}{0.5,0.5,0.5}
\definecolor{mymauve}{rgb}{0.58,0,0.82}

\lstset{
	basicstyle=\footnotesize,        % the size of the fonts that are used for the code
	breakatwhitespace=false,         % sets if automatic breaks should only happen at whitespace
	breaklines=false,                 % sets automatic line breaking
	captionpos=b,                    % sets the caption-position to bottom
	commentstyle=\color{mygreen},    % comment style
	extendedchars=true,              % lets you use non-ASCII characters; for 8-bits encodings only, does not work with UTF-8
	keepspaces=true,                 % keeps spaces in text, useful for keeping indentation of code (possibly needs columns=flexible)
	keywordstyle=\color{blue},       % keyword style
	language=[95]Fortran,                 % the language of the code
	numbers=left,                    % where to put the line-numbers; possible values are (none, left, right)
	numbersep=5pt,                   % how far the line-numbers are from the code
	numberstyle=\tiny\color{mygray}, % the style that is used for the line-numbers
	rulecolor=\color{black},         % if not set, the frame-color may be changed on line-breaks within not-black text (e.g. comments (green here))
	showspaces=false,                % show spaces everywhere adding particular underscores; it overrides 'showstringspaces'
	showstringspaces=false,          % underline spaces within strings only
	showtabs=false,                  % show tabs within strings adding particular underscores
	stepnumber=1,                    % the step between two line-numbers. If it's 1, each line will be numbered
	stringstyle=\color{mymauve},     % string literal style
	tabsize=4,                       % sets default tabsize to 2 spaces
	title=\lstname                   % show the filename of files
}


\title{Task/Eq的介绍和使用}
\author{任广智}
\date{\today}

\begin{document}
	
	\maketitle
	
	\section{定义}
	\subsection{坐标系}
	
	\begin{enumerate}
		\item 柱坐标系 $(R,\phi,Z)$
		\item 环坐标系 $(r,\theta,\zeta)$
			$$\zeta = -\phi, \nabla\zeta = -\frac{1}{R}\pmb{e_\phi}$$
	\end{enumerate}
	
	\subsection{微分方程}
	$$ \nabla\times(\nabla\zeta\times\nabla{f})=[R^2\nabla\cdot(\frac{1}{R^2}\nabla{f})]\nabla\zeta $$
	$$  $$
	
	\subsection{平衡磁场}
	$$ \pmb{B}=\frac{1}{2\pi}[I_\theta\nabla\zeta+\nabla\zeta\times\psi_\theta] $$
	$$ \pmb{j}=\frac{1}{\mu_0}\nabla\times\pmb{B}, \pmb{j}=\frac{1}{2\pi\mu_0}[R^2\nabla\cdot\frac{1}{R^2}\nabla\psi_\theta\nabla\zeta-\nabla\zeta\times\nabla{I_\theta}] $$
	
	\subsection{磁面平均}
	
	\subsection{环向磁通和环向电流}
	
	\subsection{磁面函数}
	
	
	\subsection{Grad-Shafranov 方程}
	根据磁流体力学平衡$\pmb{j}\times\pmb{B}=\nabla{P}$,我们可以通过
	$$  
	\begin{aligned}
	\pmb{j}\times\pmb{B}
	&=\frac{1}{4\pi^2\mu_0}[R^2\nabla\cdot\frac{1}{R^2}\nabla\psi_\theta+I_\theta\frac{dI_\theta}{d\psi_\theta}]\nabla\zeta\times(\nabla\zeta\times(\nabla\zeta\times\nabla\psi_\theta))\\
	&=-\frac{1}{4\pi^2\mu_0}[\nabla\cdot\frac{1}{R^2}\nabla\psi_\theta+\frac{I_\theta}{R^2}\frac{dI_\theta}{d\psi_\theta}]\nabla\psi_\theta
	\end{aligned}
	$$
	
	$$
	\nabla{P}=\frac{dP}{d\psi_\theta}\nabla{\psi_\theta}
	$$
	得到
	$$
	\nabla\cdot\frac{1}{R^2}\nabla\psi_\theta=-4\pi^2\mu_0\frac{dP}{d\psi_\theta}-\frac{I_\theta}{R^2}\frac{dI_\theta}{d\psi_\theta}
	$$
	

	\section{运行}


	
	\section{输出}
	
	\begin{lstlisting}
    integer fileUnit,NRG,NZG,NPS,NSG,NTG,NRV
	
	real(kind=8) RR,BB,RIP
	! RR    : Plasma major radius
	! BB    : Magnetic field at center
	! RIP   : Plasma current
	integer NRGMAX,NZGMAX
	! NRGMAX: Number of horizontal mesh points in R-Z plane
	! NZGMAX: Number of vertical mesh points in R-Z plane
	real(kind=8),dimension(:),allocatable :: RG,ZG
	real(kind=8),dimension(:,:),allocatable :: PSIRZ
	
	integer NPSMAX
	! NPSMAX: Number of flux surfaces
	real(kind=8),dimension(:),allocatable :: PSIPS,PPPS,TTPS,TEPS,OMPS
	
	integer NSGMAX,NTGMAX,NUGMAX,NRMAX,NTHMAX,NSUMAX,NRVMAX,NTVMAX
	! NSGMAX: Number of radial mesh points for Grad-Shafranov eq.
	! NTGMAX: Number of poloidal mesh points for Grad-Shafranov eq.
	! NUGMAX: Number of radial mesh points for flux-average quantities
	! NPSMAX: Number of flux surfaces
	! NRMAX : Number of radial mesh points for flux coordinates
	! NTHMAX: Number of poloidal mesh points for flux coordinates
	! NSUMAX: Number of boundary points
	! NRVMAX: Number of radial mesh of surface average
	! NTVMAX: Number of poloidal mesh for surface average
	real(kind=8),dimension(:,:),allocatable :: PSI,DELPSI,HJT
	real(kind=8),dimension(:),allocatable :: PSIPNV,PSIPV,PSITV,QPV,TTV
	real(kind=8) RAXIS,ZAXIS,PSITA,PSIPA,PSI0
	
	
	real(kind=8) RA,RKAP,RDLT,RB,FRBIN
	! RA    : Plasma minor radius
	! RKAP  : Plasma shape elongation
	! RDLT  : Plasma shape triangularity
	! RB    : Wall minor radius
	! FRBIN : (RB_inside-RA)/(RB_outside-RA)
	
	real(kind=8) PJ0,PJ1,PJ2,PROFJ0,PROFJ1,PROFJ2
	! PJ0   : Current density at R=RR (main component) : Fixed to 1
	! PJ1   : Current density at R=RR (sub component)       (arb)
	! PJ2   : Current density at R=RR (sub component)       (arb)
	! PROFJ0: Current density profile parameter
	! PROFJ1: Current density profile parameter
	! PROFJ2: Current density profile parameter
	
	real(kind=8) PP0,PP1,PP2,PROFP0,PROFP1,PROFP2
	! PP0   : Plasma pressure (main component)              (MPa)
	! PP1   : Plasma pressure (sub component)               (MPa)
	! PP2   : Plasma pressure (increment within ITB)        (MPa)
	! PROFP0: Pressure profile parameter
	! PROFP1: Pressure profile parameter
	! PROFP2: Pressure profile parameter
	
	real(kind=8) PT0,PT1,PT2,PROFT0,PROFT1,PROFT2
	! PT0   : Plasma temperature (main component)           (keV)
	! PT1   : Plasma temperature (sub component)            (keV)
	! PT2   : Plasma temperature (increment within ITB)     (keV)
	! PTSEQ : Plasma temperature (at surface)               (keV)
	! PROFTP0: Temperature profile parameter
	! PROFTP1: Temperature profile parameter
	! PROFTP2: Temperature profile parameter    
	
	real(kind=8) PV0,PV1,PV2,PROFV0,PROFV1,PROFV2
	! PV0   : Toroidal rotation (main component)              (m/s)
	! PV1   : Toroidal rotation (sub component)               (m/s)
	! PV2   : Toroidal rotation (increment within ITB)        (m/s)
	! PROFV0: Velocity profile parameter
	! PROFV1: Velocity profile parameter
	! PROFV2: Velocity profile parameter
	
	real(kind=8) PROFR0,PROFR1,PROFR2
	! PROFR0: Profile parameter
	! PROFR1: Profile parameter
	! PROFR2: Profile parameter
	
	real(kind=8),dimension(:,:),allocatable :: HJTRZ
	
	open(Newunit=fileUnit, file="data_test", form='unformatted')
	
	read(fileUnit) RR,BB,RIP
	read(fileUnit) NRGMAX,NZGMAX
	
	allocate( RG(NRGMAX), ZG(NZGMAX) )
	allocate( PSIRZ(NRGMAX,NZGMAX) )
	
	read(fileUnit) (RG(NRG),NRG=1,NRGMAX)
	read(fileUnit) (ZG(NZG),NZG=1,NZGMAX)
	read(fileUnit) ((PSIRZ(NRG,NZG),NRG=1,NRGMAX),NZG=1,NZGMAX)
	
	read(fileUnit) NPSMAX
	
	allocate( PSIPS(NPSMAX), PPPS(NPSMAX), TTPS(NPSMAX), TEPS(NPSMAX), OMPS(NPSMAX) )
	
	read(fileUnit) (PSIPS(NPS),NPS=1,NPSMAX)
	read(fileUnit) (PPPS(NPS),NPS=1,NPSMAX)
	read(fileUnit) (TTPS(NPS),NPS=1,NPSMAX)
	read(fileUnit) (TEPS(NPS),NPS=1,NPSMAX)
	read(fileUnit) (OMPS(NPS),NPS=1,NPSMAX)
	
	read(fileUnit) NSGMAX,NTGMAX,NUGMAX,NRMAX,NTHMAX,NSUMAX,NRVMAX,NTVMAX
	
	allocate( PSI(NSGMAX,NTGMAX),DELPSI(NSGMAX,NTGMAX),HJT(NSGMAX,NTGMAX) )
	
	read(fileUnit) ((PSI(NSG,NTG),NSG=1,NSGMAX),NTG=1,NTGMAX)
	read(fileUnit) ((DELPSI(NSG,NTG),NSG=1,NSGMAX),NTG=1,NTGMAX)
	read(fileUnit) ((HJT(NSG,NTG),NSG=1,NSGMAX),NTG=1,NTGMAX)
	
	read(fileUnit) RAXIS,ZAXIS,PSITA,PSIPA,PSI0
	
	allocate( PSIPNV(NRVMAX),PSIPV(NRVMAX),PSITV(NRVMAX),QPV(NRVMAX),TTV(NRVMAX) )
	read(fileUnit) (PSIPNV(NRV),NRV=1,NRVMAX)
	read(fileUnit) (PSIPV(NRV),NRV=1,NRVMAX)
	read(fileUnit) (PSITV(NRV),NRV=1,NRVMAX)
	read(fileUnit) (QPV(NRV),NRV=1,NRVMAX)
	read(fileUnit) (TTV(NRV),NRV=1,NRVMAX)
	
	read(fileUnit) RA,RKAP,RDLT,RB,FRBIN
	read(fileUnit) PJ0,PJ1,PJ2,PROFJ0,PROFJ1,PROFJ2
	read(fileUnit) PP0,PP1,PP2,PROFP0,PROFP1,PROFP2
	read(fileUnit) PT0,PT1,PT2,PROFT0,PROFT1,PROFT2
	read(fileUnit) PV0,PV1,PV2,PROFV0,PROFV1,PROFV2
	read(fileUnit) PROFR0,PROFR1,PROFR2
	
	allocate(HJTRZ(NRGMAX,NZGMAX))
	
	read(fileUnit) ((HJTRZ(NRG,NZG),NRG=1,NRGMAX),NZG=1,NZGMAX)
	
	close(fileUnit)
	\end{lstlisting}
	
	
\section{EQMESH}
	$(\sigma,\theta)$网格的划分
	\begin{itemize}
		\item{NSGMAX}: $\sigma$径向网格数, $N_\sigma$ 
		\item{NTGMAX}: $\theta$极向网格数, $N_\sigma$
		\item DSG: $\Delta{\sigma}$, DTG: $\Delta{\theta}$  
		\item SIGM(1:NSGMAX): $\sigma_{i+1/2}$
		\item SIGG(1:NSGMAX+1): $\sigma_i$
		\item THGM(1:NTGMAX): $\theta_{i+1/2}$
		\item THGG(1:NTGMAX+1): $\theta_i$
	\end{itemize}
	
	
\section{EQPSIN}
	初始分布
	\begin{itemize}
		\item{RAXIS=RR}:  $R_{axis} = R_0$
		\item{ZAXIS=0}: $Z_{axis} = 0$
	
	\item 	(MDLEQF=4,9)
			$$\psi_{\zeta{a}}=\pi\kappa{a^2}B_0$$
			$$\psi_{\theta{a}}=2\psi_{\zeta{a}}/q_a$$
			$$\psi_0 = -\psi_{\theta{a}}$$
			(MDLEQF=0-3,5-8)
			$$ ... $$
			
	\begin{lstlisting}
   	IF(MOD(MDLEQF,5).EQ.4) THEN
		PSITA=PI*RKAP*RA**2*BB
		PSIPA=PSITA*2/QA
		PSI0=-PSIPA
	ELSE
		PSI0=-0.5D0*RMU0*RIP*1.D6*RR
		PSIPA=-PSI0
		PSITA=PI*RKAP*RA**2*BB
	ENDIF
	\end{lstlisting}
	
	\item NRVMAX: 磁面数 
		$$ \hat{\psi_\theta(\rho)} = \rho^2 $$
		$$ \psi_\theta(\rho) = \psi_{\theta{a}}\rho^2 $$
		$$ \psi_\zeta(\rho) = \psi_{\zeta{a}}\rho^2 $$
		$$ q(\rho) = \psi_{\zeta{a}}/\psi_{\theta{a}} $$
		$$ I_\theta(\rho) = 2\pi R_0 B_0 $$
	\begin{lstlisting}
	 	DRHO=1.D0/(NRVMAX-1)
		DO NRV=1,NRVMAX
			RHOL=(NRV-1)*DRHO
			PSIPNV(NRV)=RHOL*RHOL
			PSIPV(NRV)=PSIPA*RHOL*RHOL
			PSITV(NRV)=PSITA*RHOL*RHOL
			QPV(NRV)=PSITA/PSIPA
			TTV(NRV)=2.D0*PI*RR*BB
		ENDDO
	\end{lstlisting}
		
		$$ \psi(\sigma,\theta)=\psi_0(1-\sigma^2) $$
		$$ \delta\psi(\sigma,\theta)=0 $$	
	\begin{lstlisting}
		DO NSG=1,NSGMAX
			DO NTG=1,NTGMAX
				PSI(NTG,NSG)=PSI0*(1.D0-SIGM(NSG)*SIGM(NSG))
				DELPSI(NTG,NSG)=0.D0
				HJT(NTG,NSG)=0.D0
			ENDDO
		ENDDO
	\end{lstlisting}
	
	\end{itemize}
	

\section{EQDEFB}
	\begin{itemize}
		\item RHOM(1:NTGMAX): $\rho(\theta_{j+1/2})$
		\item RHOG(1:NTGMAX+1): $\rho(\theta_{j})$
		\item RMG(1:NSGMAX,1;NTGMAX+1): $R_{i+1/2,j}$
		\item RGM(1:NSGMAX+1,1;NTGMAX): $R_{i,j+1/2}$
		\item RMM(1:NSGMAX,1;NTGMAX): 	$R_{i+1/2,j+1/2}$
		\item ZMM(1:NSGMAX,1;NTGMAX):   $Z_{i+1/2,j+1/2}$
	\end{itemize}	
	
	
	
	
	
	
	
	
\end{document}