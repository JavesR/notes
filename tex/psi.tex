\documentclass[11pt,a4paper]{article}
\usepackage{amsmath}
\usepackage{amssymb}
\usepackage{graphicx}
\usepackage{subfigure}
\usepackage{float}
\usepackage{xeCJK}
\usepackage{geometry}
\geometry{left=2.0cm,right=2.0cm,top=2.0cm,bottom=2.0cm}

\title{磁面坐标下朗道流体程序的开发}
\author{任广智}
\date{\today}


\begin{document}
	
\maketitle
	
\section{初始设置}	
\subsection{坐标系的建立} 
\subsection{方程变换} 
首先是$V_E\cdot\nabla$,这里考虑 $B_T \gg B_p$的近似,即$\pmb{B}=\pmb{B_T}=g(\psi)\nabla\phi$。但是这种假设并不是真正考虑了$k_\parallel\ll k_\perp$的结果,这一点在field aligned的坐标系下比较容易做到。所以如果是三维有限差分的话,需要考虑上$B_p$的贡献。

$$
\begin{aligned}
V_E\cdot\nabla f
&= -\frac{\nabla\Phi\times\pmb{B_T}}{B^2_0}\cdot\nabla f \\  
&= \frac{1}{B_0^2}\nabla\Phi\times\nabla f\cdot g(\psi)\nabla\phi \\ 
&= \frac{g}{JB^2_0}(\Phi_\psi f_\theta - \Phi_\theta f_\psi) \\
&= \frac{g}{JB^2_0}[\Phi,f] 
\end{aligned}    
$$	
	
$$
\begin{aligned}
V_E\cdot\nabla f_0(\psi)
&= -\frac{\nabla\Phi\times\pmb{B_T}}{B^2_0}\cdot\nabla f_0 \\
&= -\frac{g}{JB^2_0} \Phi_\theta f_\psi
\end{aligned}
$$
	
平行微分算符$\nabla_\parallel$
$$ 
\nabla_\parallel f = \pmb{b_0}\cdot\nabla f = -\frac{\Psi'J^{-1}}{B_0}(f_\theta + qf_\phi)  
$$
	
拉普拉斯算符$\nabla^2_\perp$,目前使用的一些基本算湍流输运的流体方程中,我们只使用了垂直磁场方向的laplace算符,同上边的对流项,这里的垂直不并不是严格意义上的垂直,而是在$(\psi,\theta)$平面的算符,严格的计算也会加入后续比较,即$\nabla^2_\perp f= \nabla^2 f - \nabla^2_\parallel f$,暂时如下计算:
$$
\begin{aligned}
\nabla_\perp^2 f  
&= \nabla\cdot(f_\psi \nabla\psi + f_\theta\nabla\theta) \\
&= \nabla\cdot[ (f_\psi J_{11} + f_\theta J_{21}) \nabla\theta\times\nabla\phi\mathcal{J} + (f_\psi J_{12}+f_\theta J_{22}) \nabla\phi\times\nabla\psi\mathcal{J} ] \\
&= \frac{1}{\mathcal{J}} 
[ \frac{\partial}{\partial_\psi}\mathcal{J}(J_{11}f_\psi + J_{21}f_\theta) 
+\frac{\partial}{\partial_\theta}\mathcal{J}(J_{12}f_\psi + J_{22}f_\theta) ]  
\end{aligned}
$$

磁曲率项
$$
\begin{aligned}
\nabla\cdot{\pmb{V_E}} 
&= \nabla(\frac{1}{B_0^2})\cdot(\pmb{B_0}\times\nabla\Phi) \\
&= \frac{2}{B_0^3}\nabla{B_0}\cdot(\nabla\Phi\times\pmb{B_0}) \\
&\approx \frac{2}{B_0^2}\nabla{B_0}\times\nabla\Phi\cdot\pmb{B_T}    \\
&= \frac{2g}{\mathcal{J}B_0^3}[B_0,\Phi]  
\end{aligned}
$$

\section{简单的测试}


\end{document}